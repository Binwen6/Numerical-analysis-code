% !TEX program = xelatex

\documentclass{ctexart}
\usepackage{listings}
\usepackage{mathtools}
\usepackage{ctex}
\usepackage{xcolor}
\lstset{
    basicstyle          =   \sffamily,          % 基本代码风格
    keywordstyle        =   \bfseries,          % 关键字风格
    commentstyle        =   \rmfamily\itshape,  % 注释的风格,斜体
    stringstyle         =   \ttfamily,          % 字符串风格
    flexiblecolumns,                            % 别问为什么,加上这个
    breaklines = true,  %自动折行
    numbers             =   left,               % 行号的位置在左边
    numberstyle         =   \zihao{-5}\ttfamily, % 行号的样式,小五号,tt等宽字体
    showstringspaces    =   false,
    captionpos          =   t,      % 这段代码的名字所呈现的位置,t指的是top上面
    frame               =   lrtb,   % 显示边框
}

\usepackage{amsmath,amsthm,amssymb}
\usepackage{autobreak}
\usepackage{hyperref}
\usepackage{booktabs}
\usepackage{geometry}
\usepackage{appendix}
\usepackage{multicol}
\usepackage{subfig,graphicx}
\usepackage{listings}
\newtheorem{attention}{注意}
\newtheorem{definition}{定义}[section]
\newtheorem{theorem}{定理}[section]
\newtheorem{inference}{推论}[section]
\newtheorem{xingzhi}{性质}
\newcommand{\upcite}[1]{\textsuperscript{\textsuperscript{\cite{#1}}}}
\newcommand*{\dif}{\mathop{}\!\mathrm{d}}
\def\degree{${}^{\circ}$}
%%%%%%%%%%%%%%%%%%%%%%%
%  设置水印
%%%%%%%%%%%%%%%%%%%%%%%
%\usepackage{draftwatermark}         % 所有页加水印
%\usepackage[firstpage]{draftwatermark} % 只有第一页加水印
% \SetWatermarkText{Water-Mark}           % 设置水印内容
% \SetWatermarkText{\includegraphics{fig/ZJDX-WaterMark.eps}}         % 设置水印logo
% \SetWatermarkLightness{0.9}             % 设置水印透明度 0-1
% \SetWatermarkScale{1}                   % 设置水印大小 0-1

\author{张阳}
\date{\today}
\title{数值常微分}
\numberwithin{equation}{section}
\geometry{left=2.0cm, right=2.0cm, top=2.5cm, bottom=2.5cm}

%\punctstyle{quanjiao}全角式,所有标点全角。例如,“标点挤压”。又如《标点符号用法》。 
%\punctstyle{banjiao}半角式,所有标点半角。例如,“标点挤压”。又如《标点符号用法》。 
%\punctstyle{kaiming}开明式,部分的标点半角。例如,“标点挤压”。又如《标点符号用法》。 
%\punctstyle{hangmobianjiao}行末半角式,仅行末挤压。例如,“标点挤压”。又如《标点符号用法》。 
%\punctstyle{plain}无格式,只有禁则,无挤压。例如,“标点挤压”。又如《标点符号用法》。

\begin{document}
\maketitle
\tableofcontents
\newpage
\section{各阶数值微分}
\subsection{一阶微分}
\subsubsection{向前差分}
\[u_{F}^{\prime}(x)=\dfrac{u(x+h)-u(x)}{h}-\dfrac{h}{2} u^{\prime \prime}(x+\xi)\]
\subsubsection{向后差分}
\[u_{F}^{\prime}(x)=\dfrac{u(x+h)-u(x)}{h}-\dfrac{h}{2} u^{\prime \prime}(x+\xi)\]
\subsubsection{中心差分}
\[u_{C}^{\prime}(x)=\dfrac{u(x+h)-u(x-h)}{2 h}-\dfrac{h^{2}}{6} u^{(3)}(x+\xi)\]
\subsection{二阶微分}
\[u^{\prime \prime}(x)=\dfrac{u(x+h)-2 u(x)+u(x-h)}{h^{2}}+\cdots\]

\section{第一题}
\subsection{题目}
\[y^{\prime \prime}+10 y=0, x \in(0,1) ; y(0)=0, y(1)=1\]
\subsection{数值解}
\begin{equation}
    \left\{
        \begin{array}{l}
            n = 100,h=\dfrac{1}{100},y_0=0,y_n=0\\
            \dfrac{y_{i+1}-2y_{i}+y_{i-1}}{h^2}+10y_i=0,i = 1,2,\cdots,n-1
        \end{array}
            \right.
\end{equation}
写成矩阵形式
\begin{equation}
    \begin{bmatrix}
        1 & 0 & 0 & 0 & \cdots & 0 & 0\\
        1 & (10h^2-2) & 1 & 0 & \cdots & 0 & 0\\
        0 & 1 & (10h^2-2) & 1 & \cdots & 0 & 0\\
        \vdots & \vdots & \vdots & \vdots &  & 0 & \vdots\\
        0 & 0 & 0 & 0 & \cdots & (10h^2-2) & 1\\
        0 & 0 & 0 & 0 & \cdots & 0 & 1\\
    \end{bmatrix}
    \begin{bmatrix}
        y_0\\
        y_1\\
        y_2\\
        \vdots\\
        y_{n-1}\\
        y_n
    \end{bmatrix}
    =
    \begin{bmatrix}
        0\\
        0\\
        0\\
        \vdots\\
        0\\
        0
    \end{bmatrix}
\end{equation}
\section{第二题}
\subsection{题目}
\[y^{\prime \prime}+400 y=40 \cos (20 x), x \in(0,1) ; y(0)=0, y(1)=0\]
\subsection{数值解}
\begin{equation}
    \left\{
        \begin{array}{l}
            n = 100,h=\dfrac{1}{100},y_0=0,y_n=0\\
            \dfrac{y_{i+1}-2y_{i}+y_{i-1}}{h^2}+400y_i=40\cos(20x_i),i = 1,2,\cdots,n-1
        \end{array}
            \right.
\end{equation}
写成矩阵形式
\begin{equation}
    \begin{bmatrix}
        1 & 0 & 0 & 0 & \cdots & 0 & 0\\
        1 & (400h^2-2) & 1 & 0 & \cdots & 0 & 0\\
        0 & 1 & (400h^2-2) & 1 & \cdots & 0 & 0\\
        \vdots & \vdots & \vdots & \vdots &  & 0 & \vdots\\
        0 & 0 & 0 & 0 & \cdots & (400h^2-2) & 1\\
        0 & 0 & 0 & 0 & \cdots & 0 & 1\\
    \end{bmatrix}
    \begin{bmatrix}
        y_0\\
        y_1\\
        y_2\\
        \vdots\\
        y_{n-1}\\
        y_n
    \end{bmatrix}
    =
    \begin{bmatrix}
        0\\
        h^240\cos(20x_1)\\
        h^240\cos(20x_2)\\
        \vdots\\
        h^240\cos(20x_{n-1})\\
        1
    \end{bmatrix}
\end{equation}
\section{第三题}
\subsection{题目}
\[y^{\prime \prime}+100 x y=0, x \in(0,1) ; y^{\prime}(0)=0, y(1)=1\] 
\subsection{数值解}
\begin{equation}
    \left\{
        \begin{array}{l}
            n = 100,h=\dfrac{1}{100},y_1=1\\
            \dfrac{y_1-y_0}{h}=0\\
            \dfrac{y_{i+1}-2y_{i}+y_{i-1}}{h^2}+100x_iy_i=0,i = 1,2,\cdots,n-1
        \end{array}
            \right.
\end{equation}
写成矩阵形式
\begin{equation}
    \begin{bmatrix}
        1 & -1 & 0 & 0 & \cdots & 0 & 0\\
        1 & (100x_1h^2-2) & 1 & 0 & \cdots & 0 & 0\\
        0 & 1 & (100x_2h^2-2) & 1 & \cdots & 0 & 0\\
        \vdots & \vdots & \vdots & \vdots &  & 0 & \vdots\\
        0 & 0 & 0 & 0 & \cdots & (100x_{n-1}h^2-2) & 1\\
        0 & 0 & 0 & 0 & \cdots & 0 & 1\\
    \end{bmatrix}
    \begin{bmatrix}
        y_0\\
        y_1\\
        y_2\\
        \vdots\\
        y_{n-1}\\
        y_n
    \end{bmatrix}
    =
    \begin{bmatrix}
        0\\
        0\\
        0\\
        \vdots\\
        0\\
        1
    \end{bmatrix}
\end{equation}
\section{第四题}
\subsection{题目}
\[y^{\prime \prime}+x y^{\prime}-2 y=2, x \in(0,1) ; y(0)=0, y(1)=1\] 

\subsection{数值解}
\begin{equation}
    \left\{
        \begin{array}{l}
            n = 100,h=\dfrac{1}{100},y_0=0,y_1=1\\
            \dfrac{y_{i+1}-2y_{i}+y_{i-1}}{h^2}+x_i\dfrac{y_{i+1}-y_{i-1}}{2h}-2y_i=2,i = 1,2,\cdots,n-1
        \end{array}
            \right.
\end{equation}
写成矩阵形式
\begin{equation}
    \begin{bmatrix}
        1 & 0 & 0 & 0 & \cdots & 0 & 0\\
        (2-x_1h) & -(4+4h^2) & (2+x_1h) & 0 & \cdots & 0 & 0\\
        0 & (2-x_2h) & -(4+4h^2) & (2+x_2h) & \cdots & 0 & 0\\
        \vdots & \vdots & \vdots & \vdots &  & 0 & \vdots\\
        0 & 0 & 0 & 0 & \cdots & -(4+4h^2) & (2+x_{n-1}h)\\
        0 & 0 & 0 & 0 & \cdots & 0 & 1\\
    \end{bmatrix}
    \begin{bmatrix}
        y_0\\
        y_1\\
        y_2\\
        \vdots\\
        y_{n-1}\\
        y_n
    \end{bmatrix}
    =
    \begin{bmatrix}
        0\\
        4h^2\\
        4h^2\\
        \vdots\\
        4h^2\\
        1
    \end{bmatrix}
\end{equation}
\section{第五题}
\subsection{题目}
\[y^{\prime \prime}+x^{3} y^{\prime}-3 x^{2} y=6 x, x \in(0,1) ; y(0)=0, y(1)=1 \]
\subsection{数值解}
\begin{equation}
    \left\{
        \begin{array}{l}
            n = 100,h=\dfrac{1}{100},y_0=0,y_1=1\\
            \dfrac{y_{i+1}-2y_{i}+y_{i-1}}{h^2}+x_i^3\dfrac{y_{i+1}-y_{i-1}}{2h}-3x_i^3y_i=6x_i,i = 1,2,\cdots,n-1
        \end{array}
            \right.
\end{equation}
即
\begin{equation}
    \left\{
        \begin{array}{l}
            n = 100,h=\dfrac{1}{100},y_0=0,y_1=1\\
            (2-x_i^3h)y_{i-1}-(4+6h^2x_i^3)y_i+(2+x_i^3h)y_{i+1}=12x_ih^2,i = 1,2,\cdots,n-1
        \end{array}
            \right.
\end{equation}
写成矩阵形式
\begin{equation}
    \begin{bmatrix}
        1 & 0 & 0 & 0 & \cdots & 0 & 0\\
        (2-x_1^3h) & -(4+6h^2x_1^3) & (2+x_1^3h) & 0 & \cdots & 0 & 0\\
        0 & (2-x_2^3h) & -(4+6h^2x_2^3) & (2+x_2^3h) & \cdots & 0 & 0\\
        \vdots & \vdots & \vdots & \vdots &  & 0 & \vdots\\
        0 & 0 & 0 & 0 & \cdots & -(4+6h^2x_{n-1}^3) & (2+x_{n-1}^3h)\\
        0 & 0 & 0 & 0 & \cdots & 0 & 1\\
    \end{bmatrix}
    \begin{bmatrix}
        y_0\\
        y_1\\
        y_2\\
        \vdots\\
        y_{n-1}\\
        y_n
    \end{bmatrix}
    =
    \begin{bmatrix}
        0\\
        12x_1h^2\\
        12x_2h^2\\
        \vdots\\
        12x_{n-1}h^2\\
        1
    \end{bmatrix}
\end{equation}
\section{第六题}
\subsection{题目}
\[y^{\prime \prime}+y=\tan (x), x \in(0,1) ; y(0)=0, y(1)=1 \]
\subsection{数值解}
\begin{equation}
    \left\{
        \begin{array}{l}
            n = 100,h=\dfrac{1}{100},y_0=0,y_n=1\\
            \dfrac{y_{i+1}-2y_{i}+y_{i-1}}{h^2}+y_i=\tan(x_i),i = 1,2,\cdots,n-1
        \end{array}
            \right.
\end{equation}
写成矩阵形式
\begin{equation}
    \begin{bmatrix}
        1 & 0 & 0 & 0 & \cdots & 0 & 0\\
        1 & (h^2-2) & 1 & 0 & \cdots & 0 & 0\\
        0 & 1 & (h^2-2) & 1 & \cdots & 0 & 0\\
        \vdots & \vdots & \vdots & \vdots &  & 0 & \vdots\\
        0 & 0 & 0 & 0 & \cdots & (h^2-2) & 1\\
        0 & 0 & 0 & 0 & \cdots & 0 & 1\\
    \end{bmatrix}
    \begin{bmatrix}
        y_0\\
        y_1\\
        y_2\\
        \vdots\\
        y_{n-1}\\
        y_n
    \end{bmatrix}
    =
    \begin{bmatrix}
        0\\
        h^2\tan x_1\\
        h^2\tan x_2\\
        \vdots\\
        h^2\tan x_{n-1}\\
        1
    \end{bmatrix}
\end{equation}
\section{第七题}
\subsection{题目}
\[y^{\prime \prime}+100 x^{2} y^{\prime}-x y=0, x \in(0,1) ; y(0)=0, y(1)=1 \]
\subsection{数值解}
\begin{equation}
    \left\{
        \begin{array}{l}
            n = 100,h=\dfrac{1}{100},y_0=0,y_1=1\\
            \dfrac{y_{i+1}-2y_{i}+y_{i-1}}{h^2}+100x_i^2\dfrac{y_{i+1}-y_{i-1}}{2h}-x_iy_i=0,i = 1,2,\cdots,n-1
        \end{array}
            \right.
\end{equation}
即
\begin{equation}
    \left\{
        \begin{array}{l}
            n = 100,h=\dfrac{1}{100},y_0=0,y_1=1\\
            (2-100x_i^2h)y_{i-1}-(4+2h^2x_i)y_i+(2+100x_i^2h)y_{i+1}=0,i = 1,2,\cdots,n-1
        \end{array}
            \right.
\end{equation}
写成矩阵形式
\begin{equation}
    \begin{bmatrix}
        1 & 0 & 0 & 0 & \cdots & 0 & 0\\
        (2-100x_1^2h) & -(4+2h^2x_1) & (2+100x_1^2h) & 0 & \cdots & 0 & 0\\
        0 & (2-100x_2^2h) & -(4+2h^2x_2) & (2+100x_2^2h) & \cdots & 0 & 0\\
        \vdots & \vdots & \vdots & \vdots &  & 0 & \vdots\\
        0 & 0 & 0 & 0 & \cdots & -(4+2h^2x_{i-1}) & (2+100x_{n-1}^2h)\\
        0 & 0 & 0 & 0 & \cdots & 0 & 1\\
    \end{bmatrix}
    \begin{bmatrix}
        y_0\\
        y_1\\
        y_2\\
        \vdots\\
        y_{n-1}\\
        y_n
    \end{bmatrix}
    =
    \begin{bmatrix}
        0\\
        0\\
        0\\
        \vdots\\
        0\\
        1
    \end{bmatrix}
\end{equation}
\section{第八题}
\subsection{题目}
\[y^{\prime \prime}+2 x^{2} y^{\prime}-2 x y=0, x \in(0,1) ; y(0)=0, y(1)=1\]
\subsection{数值解}
\begin{equation}
    \left\{
        \begin{array}{l}
            n = 100,h=\dfrac{1}{100},y_0=0,y_1=1\\
            \dfrac{y_{i+1}-2y_{i}+y_{i-1}}{h^2}+2x_i^2\dfrac{y_{i+1}-y_{i-1}}{2h}-2x_iy_i=0,i = 1,2,\cdots,n-1
        \end{array}
            \right.
\end{equation}
即
\begin{equation}
    \left\{
        \begin{array}{l}
            n = 100,h=\dfrac{1}{100},y_0=0,y_1=1\\
            (2-2x_i^2h)y_{i-1}-(4+4h^2x_i)y_i+(2+2x_i^2h)y_{i+1}=0,i = 1,2,\cdots,n-1
        \end{array}
            \right.
\end{equation}
写成矩阵形式
\begin{equation}
    \begin{bmatrix}
        1 & 0 & 0 & 0 & \cdots & 0 & 0\\
        (2-2x_1^2h) & -(4+4h^2x_1) & (2+2x_1^2h) & 0 & \cdots & 0 & 0\\
        0 & (2-2x_2^2h) & -(4+4h^2x_2) & (2+2x_2^2h) & \cdots & 0 & 0\\
        \vdots & \vdots & \vdots & \vdots &  & 0 & \vdots\\
        0 & 0 & 0 & 0 & \cdots & -(4+4h^2x_{n-1}) & (2+2x_{n-1}^2h)\\
        0 & 0 & 0 & 0 & \cdots & 0 & 1\\
    \end{bmatrix}
    \begin{bmatrix}
        y_0\\
        y_1\\
        y_2\\
        \vdots\\
        y_{n-1}\\
        y_n
    \end{bmatrix}
    =
    \begin{bmatrix}
        0\\
        0\\
        0\\
        \vdots\\
        0\\
        1
    \end{bmatrix}
\end{equation}
\section{误差分析}
有限差分法的潜在的误差来源是中心差分公式的截断误差,以及在求解方程组时带来的误差。此处我们调用的是MATLAB内置的求解方程组的算法,精度较高。因此,截断误差占优,误差是 $O(h^2)$,因而我们期望随着子区间 $n+1$ 升高,误差降低为 $O(h^{2})$.

我们对于问题7测试了这种方差,图xx显示了最大误差对于不同 $n$ 取值对应的解的误差 $E$ 的量级。在 log-log 图中,误差作为 n 的函数,其本质上是一条斜率为 -2 的直线,意味着,$\lg E \approx \alpha + b\lg n$,其中 $b=-2$;换句话说,与我们预期一致,误差为 $E\approx Kn^{-2}$
\end{document}